\inputencoding{utf8}
\HeaderA{evaluate}{Do evaluation.}{evaluate}
%
\begin{Description}\relax
Do evaluation for chromosomes in population by calculating model selection criterion.
\end{Description}
%
\begin{Usage}
\begin{verbatim}
evaluate(pop, model_data, model = "lm", glm_family = NULL,
  criterion = "AIC", do_parallel = FALSE)
\end{verbatim}
\end{Usage}
%
\begin{Arguments}
\begin{ldescription}
\item[\code{pop}] Matrix of population of chromosomes.

\item[\code{model\_data}] Object of class \code{model\_data}.

\item[\code{model}] Character; "lm" (default) or "glm"; Linear model or
generalized linear model.

\item[\code{glm\_family}] Character if \code{model} is "glm", \code{NULL} otherwise;
"binomial", "gaussian" (default), "Gamma", "inverse.gaussian", "poisson", "quasi",
"quasibinomial", "quasipoisson"; A family function that gives the error
distribution and link function to be used in the model.

\item[\code{criterion}] "AIC" (default) or "BIC"; Criterion to be minimized.

\item[\code{do\_parallel}] Logical; Default \code{FALSE}; Do in parallel?
\end{ldescription}
\end{Arguments}
%
\begin{Value}
Numeric vector; Evaluation values for all chromosomes
in the current generation.
\end{Value}
