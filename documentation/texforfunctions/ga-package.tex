\inputencoding{utf8}
\HeaderA{GA-package}{Genetic algorithm for model variable selection}{GA.Rdash.package}
\keyword{package}{GA-package}
%
\begin{Description}\relax
Final project for Statistics 243. An R package that implements a genetic algorithm for variable selection in linear and GLM problems.
\end{Description}
%
\begin{Details}\relax

\Tabular{ll}{
Package: & GA-package\\{}
Type: & Package\\{}
Version: & 1.0\\{}
Date: & 2014-12-13\\{}
}
\end{Details}
%
\begin{Author}\relax
Eddie Buehler, Yang Hu, JR New
\end{Author}
%
\begin{References}\relax
G. Givens and J. Hoeting. \strong{Computational Statistics, 2nd ed.} (2012).
\end{References}
%
\begin{SeeAlso}\relax
https://github.com/jrnew/genetic-algo
\end{SeeAlso}
%
\begin{Examples}
\begin{ExampleCode}
# Select regression variables for airquality data using lm model and AIC criterion
ga <- select(data = airquality,
             yvar = "Ozone",
             xvars = NULL,
             model = "lm",
             criterion = "AIC",
             pop_size = 100L,
             method_select = "rank",
             method_recombine = "onepoint",
             prob_recombine = 0.6,
             prob_mutate = 0.01,
             num_max_iterations = 100L,
             seed = 123,
             do_parallel = FALSE)

# With a user-defined model evaluation criterion function
rsquared <- function(lm) {
  mod <- summary(lm)
  return(-mod\$r.squared)
}
ga <- select(data = airquality,
             yvar = "Ozone",
             model = "lm",
             criterion = "rsquared",
             criterion_function = rsquared)
\end{ExampleCode}
\end{Examples}
