\documentclass[11pt]{article}
\usepackage[utf8]{inputenc}
\usepackage{color, hyperref, amsmath, graphicx, Rd}

\setlength{\oddsidemargin}{0in} \setlength{\textwidth}{6.5in}
\setlength{\topmargin}{0in} \setlength{\textheight}{8.5in}
\setlength{\textwidth}{6.5in} \setlength{\textheight}{8.5in}

\title{STAT 243: Software Manual for Model Selection with\\ 
	Genetic Algorithms using \texttt{ga}}
\author{Eddie Buehler, Yang Hu \& Jin Rou New\\
	University of California, Berkeley}

\date{Version 1.0, \today}

\setlength\parindent{0pt}
\setlength{\parskip}{\baselineskip}% to get space between paragraphs


\begin{document}
\maketitle

\section{Introduction}
A genetic algorithm has the following steps: 
\begin{enumerate}
	\item Calculate fitness of chromosomes.
	\item Select chromosomes to form a mating pool based on their fitness.
	\item Recombine parent chromosomes from the mating pool.
	\item Apply mutation to produce the resulting generation of chromosomes.
\end{enumerate}

\section{Code}

\begin{verbatim}
ga <- select_model(data = data, yvar = "y", xvars = NULL,
 model = "lm", glm_family = NULL, criterion = "AIC",
 pop_size = 100, method_select = "rank",
 method_recombine = "onepoint", prob_recombine = 0.6,
 prob_mutate = 0.01, num_max_iterations = 100,
 seed = 123, do_parallel = FALSE)
\end{verbatim}



The result of this function is an object of \texttt{ga} class that contains the results
\subsection{reproduce function}

\subsection{reproduce function}

\section{Testing} 

\section{Contributions} 

\section{Appendix}
\input{texforfunctions/evaluate.tex} 
 \inputencoding{utf8}
\HeaderA{evaluate\_once}{Do evaluation once.}{evaluate.Rul.once}
%
\begin{Description}\relax
Do evaluation for a chromosome by calculating model selection criterion.
\end{Description}
%
\begin{Usage}
\begin{verbatim}
evaluate_once(model_data, xvars_select, model = "lm", glm_family = NULL,
  criterion = "AIC")
\end{verbatim}
\end{Usage}
%
\begin{Arguments}
\begin{ldescription}
\item[\code{glm\_family}] Character if \code{model} is "glm", \code{NULL} otherwise;
"binomial", "gaussian" (default), "Gamma", "inverse.gaussian", "poisson", "quasi",
"quasibinomial", "quasipoisson"; A family function that gives the error
distribution and link function to be used in the model.

\item[\code{model\_data;}] Object of class \code{model\_data}.

\item[\code{xvars\_select;}] Logical vector;

\item[\code{model;}] Character; "lm" (default) or "glm"; Linear model or generalized linear model.

\item[\code{criterion;}] "AIC" (default) or "BIC"; AIC or BIC.
\end{ldescription}
\end{Arguments}
%
\begin{Value}
Numeric; Value of criterion.
\end{Value}
 
 \input{texforfunctions/initialize.tex} 
 \input{texforfunctions/mutate.tex} 
 \input{texforfunctions/plot_ga.tex} 
 \input{texforfunctions/process_data.tex} 
 \input{texforfunctions/recombine.tex} 
 \input{texforfunctions/recombine_once.tex} 
 \inputencoding{utf8}
\HeaderA{reproduce}{Wrapper function for reproduction stage.}{reproduce}
%
\begin{Description}\relax
Wrapper function for reproduction stage.
\end{Description}
%
\begin{Usage}
\begin{verbatim}
reproduce(ga, iteration, do_parallel = FALSE)
\end{verbatim}
\end{Usage}
%
\begin{Arguments}
\begin{ldescription}
\item[\code{ga}] Object of class \code{ga}.

\item[\code{iteration}] Iteration number.
\end{ldescription}
\end{Arguments}
%
\begin{Value}
Updated \code{ga} list object.
\end{Value}
 
 \input{texforfunctions/select.tex} 
 \input{texforfunctions/select_model.tex} 
 \input{texforfunctions/summary_ga.tex} 


\end{document}